\documentclass[12pt]{article} 
\usepackage{amssymb}
\linespread{1.25}

\begin{document} 

\title{ECO 2306 Midterm Guide\\
(With Formulas at the End)}
% \author{Julio Huato}
\date{\today}

\maketitle

NOTE: The answers in this guide may contain math errors.  If you find one, let me know right away (so I can share the correction with the rest of the class) and you'll receive extra credit toward the midterm.  (If you just report a spelling error or textual typo, you'll have my grateful appreciation.)

\vspace{16pt}

\noindent 1. The assets of XYZ, Inc. are worth \$16 million and its liabilities (in the accounting sense of the term) \$11 million.  What is XYZ's equity?

\vspace{10pt}


\texttt{XYZ,Inc.'s equity is $E = A - L = 16 - 11 = 5$ million dollars.}

\vspace{12pt}


\noindent 2. In the year, BCD Corp's revenues were \$42 million, the cost of goods sold \$22 million, operating expenses (including depreciation) \$8 million, financial expenses \$6 million, and taxes \$8 million.  What was BCD's annual net profit?

\vspace{10pt}

\texttt{BCD's net profit was $\Pi = R - C - \textrm{financial expenses} - \textrm{taxes} = 42-22-8-6-8= -2$.  BCD has a loss of \$2 million.}

\vspace{12pt}

\noindent 3. Indicate whether each of the following magnitudes is a stock or a flow.\\

\begin{center}
\begin{tabular}{l | c}
  \hline    	
  Variable & Type \\
  \hline      
Balance in savings account & $Stock$ \\
Trade deficit & $Flow$ \\
Monthly wages & $Flow$ \\
National debt & $Stock$ \\
Labor income (wages, salaries, benefits) & $Flow$ \\
Gross domestic product & $Flow$ \\
Quarterly profits & $Flow$ \\
Debt (liabilities) & $Stock$ \\
Monthly expenses in food and clothing & $Flow$ \\
Annual interest & $Flow$ \\
Individual wealth (net worth or equity) & $Stock$ \\
Household consumption spending & $Flow$ \\
Private investment spending & $Flow$ \\
Property income (profits, rents, interests) & $Flow$ \\
Government budget deficit & $Flow$ \\
Balance in checking account & $Stock$ \\
National wealth & $Stock$ \\
Annual exports & $Flow$ \\
Annual imports & $Flow$ \\
\hline
\end{tabular}
\end{center}

\vspace{12pt}

\noindent 4. The population of a country is $120$ million and its labor force is three fifths of the population.  Determine the size of its labor force.

\vspace{10pt}

\texttt{The labor force in this country is $L = (3/5) \times 120 = 72$ million people.}

\vspace{12pt}

\noindent 5. If the unemployment rate is 8 percent and the labor force is 72 million people, determine the number of people unemployed. 

\vspace{10pt}


\texttt{The unemployed are $U = u \ L = .08 \times 72 = 5.76$ million people.}

\vspace{12pt}

\noindent 6. In 2014, the GDP deflator is 122 (percentage units).  In 2015, it is 124.  Determine the GDP annual inflation rate (or growth rate of the price level). 

\vspace{10pt}


\texttt{The GDP annual inflation rate is  $\pi = \frac{P_{15}}{P_{14}} - 1 = \frac{1.24}{1.22} - 1 \approx .0164 = 1.64\%$.}

\vspace{12pt}



\noindent This year, households in the economy receive \$30 billion in labor income (wages, salaries, and benefits) and \$24 billion in property income (rents, interests, and profits).  Also this year, households pay \$18 billion in taxes and spend \$20 billion in consumption goods.  Firms' investment (their annual spending in capital stock) is \$12 billion. The government sector runs a \$2 billion deficit (i.e. $G-T = B_G = 2$).  

\vspace{10pt}

\noindent 7. Determine the total income of all households in this economy.

\vspace{10pt}

\texttt{Total income is $Y = Y_L + Y_P = 30 + 24 = 54$ billion dollars.}

\noindent 8. Determine the disposable income of households.

\vspace{10pt}

\texttt{Disposable income is $Y_D = Y - T = 54 - 18 = 36$ billion dollars.}


\vspace{12pt}

\noindent 9. Determine the annual savings by households.

\vspace{10pt}

\texttt{The annual savings by households in this economy are $S = Y_D - C = 36 - 20 = 16$ billion dollars.}

\vspace{12pt}

\noindent 10. Determine the annual net borrowing by firms.

\vspace{10pt}

\texttt{By assumption, the annual net borrowing by firms equals investment spending by firms.  That is, $I = B_B = 12$ billion dollars.}

\vspace{12pt}

\noindent 11. Determine the annual net exports (i.e. $X-Q$).

\vspace{10pt}

\texttt{Since $Y = C + I + G + (X-Q)$, then $(X-Q)= Y - C - I - G = 54 - 20 - 12 - 20 = 2$ billion dollars.}

\vspace{12pt}

\noindent 12. Considering your answer to question 11, is the rest of the world ``borrowing'' from this economy?  Or is the rest of the world ``lending'' to this economy?  Explain.  (Remember that ``borrowing by the rest of the world'' means here not only granting loans to foreigners, or purchasing their bonds, but also the purchasing their stocks and other forms of wealth.  Similarly, ``lending by the rest of the world'' is not only receiving loans from foreigners or selling them our bonds, but also selling them our stocks and other kinds of wealth.)

\vspace{10pt}

\texttt{Since $X-Q = B_{ROW}$, in net terms the rest of the world is borrowing from this economy \$2 billion.}

\vspace{12pt}

\noindent 13. List assumptions of the simple macroeconomic model reviewed in class.

\vspace{10pt}

\texttt{The following are assumptions on which the macro model reviewed in class is based:
\begin{enumerate}
\item The economy produces one single undifferentiated good (``real output'' or ``real income'' or ``real GDP'') that people can consume or invest (i.e. use to build up or expand their capital stock), use for government purposes, or export
\item The price level in the economy is constant, i.e. the level of real output or income (real GDP) and employment adjust in the short run to changes in spending without affecting the price level
\item	Consumption depends on disposable income and a host of other factors that are not explicitly analyzed (captured by the catch-all term ``autonomous consumption spending'')
\item	The factors that drive investment are not explicitly analyzed
\item	The factors that drive government spending, taxes, and exports are not explicitly analyzed.
\item Investment spending is supposed to be carried out by businesses or firms
\item	The financial system just channels the savings of households and allocates them into private business investment, government, or rest of the world
\item	Government deficits and debt are not explicitly analyzed
\item	The interest rate (and its impact on spending by households, businesses, government, and rest of the world) is not explicitly analyzed
\item No distinction is made between households that depend mainly on property income ("capitalists") and households that depend mainly on labor income ("workers"), or between their different patterns of consumption spending
\end{enumerate}
}

\vspace{12pt}

\noindent Suppose the annual level of autonomous consumption spending in the economy is \$20 billion, the marginal propensity to consume is $.6$, net taxes are \$10 billion, government spending is \$10 billion, private investment spending is \$40 billion, and the trade balance (exports net of imports) is \$0.

\vspace{10pt}

\noindent 14. Determine the equilibrium level of income (or output). 

\vspace{10pt}


\texttt{The equilibrium level of income in the economy is:}

\[Y^* = \left( \frac{1}{1 - c} \right) [\bar{C} + I + G + (X-Q)] - \left( \frac{c}{1 - c} \right) T,\]
\[Y^* = \left( \frac{1}{1 - .6} \right) [20 + 40 + 10 + 0] - \left( \frac{.6}{1 - .6} \right) (10),\]
\[Y^* = \left( \frac{1}{.4} \right) (70) - \left( \frac{.6}{.4} \right) (10), \]
\[Y^* = \left( 2.5 \right) (70) - \left( 1.5 \right) (10), \]
\[Y^* = 175 - 15, \]
\[Y^* = 160.\]

\texttt{That is, $160$ billion dollars.}

\vspace{12pt}

\noindent 15. Determine the equilibrium level of consumption spending. 

\vspace{10pt}

\texttt{The equilibrium level of consumption spending is:}

\begin{eqnarray}
C^* = \bar{C} + c \ (Y^* - T), \nonumber \\
C^* = 20 + (.6) (160-10), \nonumber \\
C^* = 20 + (.6) (150), \nonumber \\
C^* = 20 + 90, \nonumber \\
C^* = 110. \nonumber 
\end{eqnarray}

\texttt{That is, $110$ billion dollars.}

\vspace{12pt}

\noindent 16. Determine the equilibrium level of saving. 

\vspace{10pt}

\texttt{The equilibrium level of saving:}

\begin{eqnarray}
S^* = Y^* - T - C^*, \nonumber \\
S^* = 160 - 10 - 110, \nonumber \\
S^* = 40. \nonumber 
\end{eqnarray}

\texttt{That is, $40$ billion dollars.}

\vspace{12pt}

\noindent 17. Suppose the government cuts its spending by \$2 billion.  Determine the new equilibrium level of income. 

\vspace{10pt}


\texttt{The new equilibrium level of income:}

\begin{eqnarray}
Y^*_1 = Y^*_0 + \Delta Y^*, \nonumber \\
\end{eqnarray}

\noindent where $\Delta Y^* =  \left( \frac{1}{1 - c} \right) \Delta G$, with $\Delta G = -2$.  Thus, $\Delta Y^* = \left( 2.5 \right) (-2) = -5$.  Therefore:

\begin{eqnarray}
Y^*_1 = 160 - 5, \nonumber \\
Y^*_1 = 155. \nonumber \\
\end{eqnarray}

\texttt{That is, $155$ billion dollars. Another way to answer this question is to recalculate $Y^*$ with $G = 8$.  You should verify that the result is the same.
}

\vspace{12pt}


\noindent 18. Suppose that, instead of cutting its spending, the government decides to increase its taxes by \$2 billion.  Determine the new equilibrium level of income. 

\vspace{10pt}

\texttt{The new equilibrium level of income:}

\begin{eqnarray}
Y^*_1 = Y^*_0 + \Delta Y^*, \nonumber \\
\end{eqnarray}

\noindent where $\Delta Y^* =  - \left( \frac{c}{1 - c} \right) \Delta T$, with $\Delta T = 2$.  Thus, $\Delta Y^* = \left( - 1.5 \right) (2) = -3$.  Therefore:

\begin{eqnarray}
Y^*_1 = 160 - 3, \nonumber \\
Y^*_1 = 157. \nonumber \\
\end{eqnarray}

\texttt{That is, $157$ billion dollars.  Alternatively, you can recalculate $Y^*$ with $T = 12$.  Verify that the result is the same.}

\vspace{12pt}

\noindent 19. Compare the scenarios in questions 17 and 18.  What is the effect of each of these policies on the government's budget condition?  In particular, will the government run a deficit or surplus, or will it have a balanced budget?  And what can you say about the different effect on the level of the economy?

\vspace{10pt}

\texttt{In these scenarios, the effect on the government budget of lowering spending or increasing taxes is the same: a \$2-billion surplus.  However, the effects on the economy are different due to the different multipliers: A \$1-reduction in government spending shrinks the economy by \$2.50, while a \$1-increas in taxes only shrinks the economy by \$1.50.}

\section*{Formulas}

\subsection*{Stocks, flows, accounting}

\begin{eqnarray} A = L + E, \end{eqnarray}

\noindent where $A$ is assets, $L$ is liabilities, and $E$ is equity or net worth.

\begin{eqnarray} \Pi = R - C - \textrm{fin. expenses} - \textrm{taxes}, \end{eqnarray}

\noindent where $\Pi$ is profit and $R$ is revenue from sales.

\begin{eqnarray} u = U/L, \end{eqnarray}

\noindent where $u$ is the unemployment rate, $U$ is the number of unemployed members of the labor force, and $L$ is the number of people in the labor force.

\begin{eqnarray} \pi_t = \frac{P_t}{P_{t-1}} - 1,\end{eqnarray}

\noindent where $\pi_t$ is the annual inflation rate for year $t$, and $P_t$ is the price level at the end of year $t$.

\subsection*{National accounting}

\begin{eqnarray} Y = Y_L + Y_P, \end{eqnarray}

\noindent where $Y$ is total income, $Y_L$ is labor income, and $Y_P$ is property income.

\begin{eqnarray} Y_D = Y - T, \end{eqnarray}

\noindent where $Y_D$ is disposable income and $T$ are net taxes.

\begin{eqnarray} G - T = B_G,\end{eqnarray}

\noindent where $G$ is government spending and $B_G$ is the government budget deficit.  Note that if $B_G > 0$, there is a budget deficit.  If $B_G < 0$, there is a budget surplus.  If $B_G = 0$ (i.e. if $G = T$), then the government is running a balanced budget.

\begin{eqnarray} I = B_B, \end{eqnarray}

\noindent where $I$ is investment spending by businesses and $B_B$ is borrowing by businesses, which (by assumption) are equal.

\begin{eqnarray} X-Q = B_{ROW}, \end{eqnarray}

\noindent where $X$ is exports, $Q$ imports, $X-Q$ net exports, and $B_{ROW}$ borrowing by the rest of the world.  If $X-Q = B_{ROW} > 0$, then the rest of the world is ``borrowing'' from this economy.  If  $X-Q = B_{ROW} < 0$, then the rest of the world is ``lending'' to this economy.  If $X-Q = B_{ROW} = 0$, there is a balanced trade.

This is the fundamental equation in the national accounts and you should know it by heart:

\begin{eqnarray} Y = C + I + G + (X-Q).\end{eqnarray}

This equation closes the financial markets loop in our diagram:

\begin{eqnarray} S = B_G + B_B + B_{ROW}.\end{eqnarray}

Keep in mind that, while $B_G$ and $B_B$ are typically positive, $B_{ROW}$ may be positive, negative, or zero.

\subsection*{Simple macro (``Keynesian'') model}

Note that, while in national accounting all of these variables refer to historical or past values, in this section these same variables refer to ``future'' or ``planned'' levels.  So, for example $Y$ is planned output or income while $C$ is planned consumption spending by households.  To avoid verbosity, we tend not to say ``planned'' or ``future,'' but you should keep that in mind.

The consumption function:

\begin{eqnarray} C = \bar{C} + c (Y - T) = (\bar{C} - c) \ T + c \ Y, \end{eqnarray}

\noindent where $C$ is consumption spending, $\bar{C}$ autonomous (or independent of income) consumption spending, $c$ is the marginal propensity to consume (or part of each extra dollar of income that households will add to their consumption spending), and $T$ is net taxes.

The equilibrium condition:

\begin{eqnarray} Y = C + I + G + (X-Q)\end{eqnarray}

\noindent where the categories mean the same as usual, but---again---in future or forward looking terms.

The solution of the model, for the equilibrium level of (planned) output or income:

\begin{eqnarray} Y^* =  \left( \frac{1}{1 - c} \right) [\bar{C} + I + G + (X-Q)] - \left( \frac{c}{1 - c} \right) [T]. \end{eqnarray}

The equilibrium level of (planned) consumption spending:

\begin{eqnarray} C^* =  \bar{C} + c (Y^* - T). \end{eqnarray}

The equilibrium level of (planned) savings:

\begin{eqnarray} S^* =  Y^* - T - C^*. \end{eqnarray}

These are all levels.  As for what happens to $Y^*$ when fiscal policy changes are introduced:

\begin{eqnarray} \Delta Y^* = \left( \frac{1}{1 - c} \right) \Delta G, \end{eqnarray}

\noindent when $G$ changes from an initial to a final level: $\Delta G = G_1 - G_0$.  Note that $Y^*_1 = Y^*_0 + \Delta Y^*$, where $Y^*_0$ is the initial equilibrium level of output and $Y^*_1$ is the final level.

This is when taxes $T$ change:

\begin{eqnarray} \Delta Y^* = - \left( \frac{c}{1 - c} \right) \Delta T. \end{eqnarray}

Beware of the negative sign.

Although we focus on the effect of fiscal policy variables $G$ and $T$, variables the government directly control, it is also true that:

\begin{eqnarray} 
\Delta Y^* = \left( \frac{1}{1 - c} \right) \Delta \bar{C}, \\
\Delta Y^* = \left( \frac{1}{1 - c} \right) \Delta I, \\
\Delta Y^* = \left( \frac{1}{1 - c} \right) \Delta (X-Q).
\end{eqnarray}

The government tries (or should try) to influence the behavior of households and businesses at home, and households, businesses, and governments abroad, in order to make the economy expand without having to increase taxes or reduce spending, or to contract when the economy is growing too fast for safety.  I guess we could call that approach ``economic leadership'' or ``open-mouth operations.''\footnote{This is a play with the words used to refer to a common instrument of monetary policy: ``open-market operations.''}


\end{document}
